% small.tex
\documentclass{beamer}
\usetheme{default}
\usepackage{fancybox}
\title{UNIX and Shell From the Start}
\author{Paddy Corr\\\textless paddy@netsoc.tcd.ie\textgreater}

\begin{document}
\begin{frame}[plain]
  \titlepage
\end{frame}

\begin{frame}{Connecting}
\texttt{ssh} (secure shell) to one of our servers using:

\begin{itemize}
    \item PuTTY on windows with login.netsoc.tcd.ie
    \item using a terminal on unix \$ssh name@login.netsoc.tcd.ie
\end{itemize}


\end{frame}

\begin{frame}{The Basics}
If you have never worked with a UNIX shell then these next few slides are the slides to get you started.\\We will go through:
\begin{itemize}
    \item ls
    \item cd
    \item rm
    \item rmdir
    \item mkdir
    \item mv
    \item cp
\end{itemize}
But first; What are we looking at?
\end{frame}

\begin{frame}{What is this?}
\begin{center}
    \fbox{\texttt{paddy@cube:\~\$}}
\end{center}

This is your prompt. It shows your username (paddy), the hostname (cube), and the directory (\textasciitilde   : tilde means home).


\end{frame}

\begin{frame}{ls}
ls lists the contents of a directory. 
\begin{center}
    \fbox{\texttt{\$ ls -al}}
\end{center}
in this example the -al are the arguments passed to ls.
\begin{itemize}
    \item -a means all and will list hidden files in a directory
    \item -l means long listing format and will list more info about each file
\end{itemize}
output:\\
\texttt{-rw-r-----   1 paddy paddy   3047 Aug 19 22:51 .bashrc}
shows the privileges, number of files in a folder, user etc.
\end{frame}

\begin{frame}{cd}
\texttt{cd} will change directory.\\If we do \texttt{ls -a} we can se that there are folders called . and ..\\"." means the current directory and ".." means the directory that is above this one.
to change into a different directory:
\begin{center}
    \fbox{\texttt{\$cd ..} or \texttt{\$cd <foldername from ls>}}
\end{center}
\end{frame}

\begin{frame}{rm}
\texttt{rm} removes a file forever.\\
WARNING: UNIX is not like windows and does not have a recycling bin. If you rm something it will be gone forever.\\
Usage:
\begin{center}
    \fbox{\texttt{\$rm [-rvf] <filename>}}
\end{center}
\begin{itemize}
    \item -r means recursively and will delete all file in a folder and the folder
    \item -v means verbal and will list files as it deletes
    \item -f means force
\end{itemize}
\end{frame}

\begin{frame}{mkdir}
Creates one or more new folders
\begin{center}
    \fbox{\texttt{\$mkdir <filename> <filename2> <etc>}}
\end{center}
\end{frame}

\begin{frame}{mv}
Moves or renames files or directorys.\\The argument -f can be used to force.\\
To rename a file or directory:
\begin{center}
    \fbox{\texttt{\$mv file1 file2}}
\end{center}
\begin{center}
    \fbox{\texttt{\$mv directory1 directory2}}
\end{center}
Move a file to a directory:
\begin{center}
    \fbox{\texttt{\$mv file1 Directory}}
\end{center}
If the folder you are moving a file to has a file with the same name it will be deleted first. See WARNING about rm.
\end{frame}

\begin{frame}{cp}
Copy a file\\
\begin{center}
    \fbox{\texttt{\$cp file1 file2}}
\end{center}
Copy to a different directory:
\begin{center}
    \fbox{\texttt{\$cp file directory}}
\end{center}
\end{frame}

\begin{frame}{man}
\texttt{man} is your friend.\\man is the manual for commands in UNIX.\\If you want to know how to use ls for example then type : 
\begin{center}
    \fbox{\texttt{\$man ls}}
\end{center}
This will show you a list of arguments for ls along with things like the Author and usage etc.\\
\texttt{man} should be your first port of call, yes even before Google.
\end{frame}

\begin{frame}{ls}
\end{frame}

\end{document}
